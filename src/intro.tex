Wolfenstein 3D, выпущенный 5 мая 1992 года, установил такой жанр как FPS (First Person Shooter - стрелялка от первого лица). Дизайн игры, усиленный игровым движком обеспечивающем отличную 256-цветную графику, скорость, высокий фреймрейт (частота кадров), умный ИИ, чёткие звуковые эффекты, увлекательную музыку, всё это получило всеобщее признание. В течение года игра разошлась в количестве более 100 000 экземпляров\footnote{Игра распространялась через модель shareware.}, что принесло всеобщее признание и немного удачи команде разработчиков: id Software.\\
\par
\begin{figure}[H]
\centering
\fullimage{action_packed.png}
\end{figure}
\par
Однако, фанаты не остановились на простом прохождении игры. Движимые духом созидания, в желании создать собственных персонажей и карты, они начали исследовать и разбирать код игры. В течение нескольких месяцев форматы игры были полностью разобраны, и были выпущены различные моды\footnote{МОДифицированная версия.} с изменённой графикой, звуковыми эффектами, музыкой и картами. Однако, ядро игры и секрет ее быстрой работы оставались не раскрытыми.\\
\\
Это держалось в секрете по вполне очевидной причине: мощный движок - важный актив для игровой компании. В качестве средства для победы над конкурентами, хорошая деловая практика - держать других программистов в неведении. Это позволяет сохранить технологическое преимущество, создавать лучшие игры и получать больше прибыли.\\
\\
Тем не менее, некоторые люди в id Software считали по-другому. Вместо того, чтобы придерживаться здравого смысла, они хотели принять энтузиазм игроков и полностью открыть исходный код для публики. После долгих внутренних дискуссий id Software делает немыслимое: 21 июля 1995 года выгружается zip-архив на\emph{ftp.idsoftware.com}, содержащий полный исходный код движка с инструкциями по его сборке\footnote{Это было не таким уж и безумием, 10 декабря 1993 года вышел Doom, сделавший Wolfenstein 3D устаревшим.}. \\
\par
 \begin{fancyquotes}
   Программирование - это игра с ненулевой суммой. Делясь знаниями с коллегами, вы не теряете их. Я счастлив делиться всем, что знаю просто из любви к программированию.\\
   \\
\textbf{Джон Кармак - программист}
 \end{fancyquotes}\\
\\
Открытый код, кроме того, что привлекает программистов, имеет еще две инетересных особенности.\\
\\
Во-первых, это увеличивает срок жизни программы даже после того, как целевое оборудование или операционная система перестают выпускаться. Имея доступ к исходникаМ, программисты могут улучшить или портировать движок на новую ОС или железо. Спустя двенадцать лет после релиза Wolfenstain 3D вы всё еще можете поиграть в него на любом процессоре, с любой памятью, с любой видекартой.\\
\\
Во-вторых, это открывает окно назад, в 1991 год. Разобрав такие сложные движки как Quake III и Doom III на \emph{fabiensanglard.net}, я подумал, что стоило бы вскольз пробежаться по движку Wolfenstain 3D, и его "простой" технологии трассировки луча. Когда я из любопытства посмотрел поглубже, меня поразило то, что я увидел, и я уже не смог остановиться в своём исследовании. Чем дальше, тем больше я осознаю, что такой целевой компьютер как IBM PC, предназначен больше для офисной работы, а не для игр. Основное его назначение - целочисленный расчёт и отображение статических изображений для обработки текстови электронных таблиц.
%Это история, которую пытается рассказать эта книга.
То, что сделала id Software\footnote{Другие компании, такие как Origin Systems и LucasArts так же творили невероятное.} в 1991 году - это не просто запрограммировала машину, они превратили инструмент, созданный для офисной работы в лучшую игровую платформу в мире.\\
\\
Но для чего такие сложности? Вообще, если у вас есть игровая компания, и вы хотите создавать видеоигры, существуют игровые консоли, ориентированные исключительно на эту задачу. Genesis, Super NES и Neo-Geo имели в своём арсенале спрайтовые движки\label{sprite_engine_ref}, которые, несмотря на ограничения по размеру и количеству спрайтов, позволяли передвигать на экране изображения просто обновляя их $(x,y)$ координаты. Они могли запросто генерировать плавную анимацию в 60 кадров в секунду, имели контроллеры, аудио-системы для звука и музыки, и были однородными (т.е., любая SNES консоль была идентично другой такой же). Если вам все же хотелось использовать персональный компьютер для игр, почему бы не взять Amiga 500, который поставлялся с сопроцессором, специально разработанным для анимации?\\
\\
Причина содержится в одном слове: кадровый буфер. Тот тип игры, который id Software хотело создать, не мог использовать спрайтовый движок, или трюки Cooper\footnote{Прозвище мощного сопроцессора на Amiga, позволяющего выполнять операции уровня вертикальной синхронизации.}. Они хотели потрясти весь игровой мир, предоставив потрясающий опыт в трёх измерениях. Для этого им нужно было обновлять весь экран, пиксель за пикселем, в кадровом буфере до того, как он отображался на мониторе.\\
\par
Чтобы отобраизть все эти пиксели, им нужен мощный ЦП, и ПК, первосходящий любую консоль на рынке. Amiga\footnote{Джимми Махер выдвинул интересную теорию в своей книге "Будущее здесь: The Commodore Amiga": Люди хотели играть в шутеры от первого лица, что архитектура Amiga не могла позволить. Эта особенность, в конечном итоге, привела к краху бесселера от Commodore.}, даже со своим сопроцессором, не могла конкурировать по мощности с ПК.
\par
\input{imgs/drawings/available_hardware_mips.tex}

 
Имея быстрый ЦП и кадровый буфер размером в 256кБ, в 1991 году ПК выглядел самым многообещающим. Однако, было три, казалось бы, непреодолимых препятствия\footnote{The title of this book could have been "The Impossible Machine".}:
\begin{itemize}
\item Видеосистема (называемая VGA), не имела двойного буфера. Не было возможности отобразить плавную анимацию без неприятных артефактов на экране, называемых "слёзы".
\item Процессор мог работать только с целочисленными операторами, но трёхмерные преобразования требуют дробные вычисления.
\item Динамик, которым по умолчанию комплектовался ПК, мог воспроизводить только квадратичные волны, и в результате лишь "бикал", не в силах воспроизвести ничегои иного.
\end{itemize}
Помимо этого, были еще серьёзные проблемы:
\begin{itemize}
\item Адресный режим ОЗУ был не плоским, а сегментированным, в результате это приводит к сложной и подверженной ошибкам арифметике указателей.
\item Пиксели на VGA не были квадратными, кадровый буфер был вытянут по вертикали.
\item Большие различия в аудио-системах. Каждая система имела свои особенности и ограничения.
\item Машина могла оперировать только 1 Мб ОЗУ, для преодоления этого барьера необходимо было использовать драйвера, каждый со своей особенностью.
\item Шина была очень медленной и обмен данными с видеопамятью был бутылочным горлышком. В частности, невозможно было переписывать весь экран с частотой более 70 кадров в секунду.
\end{itemize}

В целом, казалось, что ПЦ был обречён на выполнение скучных задач. Но кое-кто в мире не принял этого факта и использовал аппаратные средства ПК для достижения довольно неожиданных результатов. Как они это сделали, описывается в этой книге. Я разделил её на три части:
\begin{itemize}
\item Часть 2: Аппаратное обеспечение. Пять компонентов ПК из 1991.
\item Часть 3: Команда\footnote{Это техническая книга. Для изучения человеческих отношений можно прочитать книгу Дэвида Кушнера: "Masters of Doom".}. Люди передвигают границы.
\item Часть 4: Программное обеспечение. Игровой движок The Wolfenstein 3D.
\end{itemize}
\par
Впервые демонстрируя аппаратные ограничения, я надеюсь, что программисты оценят По, и то, как как оно преодолевает эти ограничения, зачастую превращая их в преимущества.\\
\pagebreak



 \bu{Trivia :} Название "Wolfenstein 3D" впервые было представлено в 1981 году на Apple II в игре "Castle Wolfenstein" Силас Уорнера.\\
\par
 \begin{fancyquotes}
   Мы названи её Wolfenstain 3-D, так как оригинальная игра Castle Wolfenstain имела сиквел Beyond Castle Wolfenstein, который был \#2. Наша версия Castle Wolfenstein юыда \#3 и она была в 3-D. Мы использовали ту же систему наименования, что и с Catacomb, Catacomb 2 и Catacomb 3-D.
 \bigskip \\
\textbf{Джон Ромеро}
 \end{fancyquotes}
 \\
\par
 \cscaledimage{1}{intro_wolf_appleii.png}{Заглавный экран "Castle Wolfenstein"}
\par
 \par Версия игры на Apple была ориентирована на скрытность (стиль Wolfenstein 3-D "Полный убой" явно отличался от оригинальной игры), при этом выделялась благодаря использованию оцифрованных голосов, что на тот момент было беспрецендентно.
 \par
 Изначально, команда считала, что им не удастся использовать имя Wolfenstein из за проблем с торговой маркой. Однако, в 1992 году они смогли найти в г.Байлморе, штат Мэрилэнд женщину, которая владела торговой маркой Wolfenstain, и купить её за \$5,000.\\
 \par


\begin{figure}[H]
\centering
\scaledimage{1}{CastleWolfensteinAppleII.png}
\caption{Оринетированный на скрытность "Castle Wolfenstein".}
\end{figure}


